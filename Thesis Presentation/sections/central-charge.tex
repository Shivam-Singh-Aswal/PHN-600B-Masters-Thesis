\section{Central Charge}


\begin{frame}[allowframebreaks]{Mathematical Definition of Central Charge}
    \textbf{Central Charge (\cite{Brown:1986nw},\cite{Hartman:2008dq}, \cite{Rathi:2024}) in 2D CFT:}
\begin{itemize}
  \item The energy-momentum tensor generates the Virasoro algebra:
  \[\left[L_m, L_n\right] = (m - n)L_{m+n} + \frac{c}{12}(m^3 - m)\delta_{m+n, 0}\]

  \item The constant \( c \) is the \textbf{central charge}.
\end{itemize}

\textbf{Gravity Perspective (Brown–Henneaux \cite{Brown:1986nw}):}
\begin{itemize}
  \item In 3D gravity with AdS\(_3\) boundary conditions:
  \[c = \frac{3\ell}{2G_3}\]
  where \( \ell \) is the AdS radius and \( G_3 \) is Newton's constant in 3D.
\end{itemize}

\textbf{2D Gravity on AdS\(_2\) \cite{Castro:2008ne} (e.g., Jackiw-Teitelboim gravity):}
\begin{itemize}
  \item With Maxwell-dilaton gravity and a twisted energy-momentum tensor:
  \[\tilde{T}_{\pm\pm} = T_{\pm\pm} \pm A \partial_\pm j_\pm\]

  the central charge becomes:
  \[c = 12kG^2Q^2\ell^4\] 

  where \( Q \) is electric charge, \( \ell \) the AdS\(_2\) radius, and \( k \) the level of the U(1) current.
\end{itemize}
\end{frame}

% Slide 2: Physical Significance
\begin{frame}[allowframebreaks]{Physical Significance of Central Charge}
\textbf{1. Holography and AdS/CFT:}
\begin{itemize}
  \item The central charge governs the density of states in a 2D CFT.
  \item In AdS/CFT, it encodes the number of degrees of freedom on the boundary.
\end{itemize}

\textbf{2. Entropy via Cardy Formula:}
\[S = 2\pi \sqrt{\frac{c}{6} \left( \Delta - \frac{c}{24} \right)}\]

\begin{itemize}
  \item This formula matches black hole entropy when \( c \) comes from asymptotic symmetries.
  \item Example: AdS\(_2\)/CFT\(_1\) — entropy from twisted CFT matches gravity result.
\end{itemize}

\textbf{3. Indicator of Anomalies:}
\begin{itemize}
  \item Central charge appears in quantum anomalies (e.g., conformal anomaly).
  \item Shows failure of classical symmetry at the quantum level.
\end{itemize}

\textbf{4. Near-Horizon Symmetries:}
\begin{itemize}
  \item Emergence of a CFT near black hole horizons with well-defined central charge.
  \item Supports holographic descriptions even in 2D (AdS\(_2\) cases).
\end{itemize}
\end{frame}

\begin{frame}[allowframebreaks]{Central Charge for AdS\(_2\) Gravity {\color{red}\scriptsize [Hartman et al.]\cite{Hartman:2008dq}}}
    
\textbf{Setup:} 2D Maxwell-dilaton gravity on AdS\(_2\) with constant electric field \( E \), AdS radius \( \ell \), and gauge field \( A_\mu \).  

\bigskip

\textbf{1. Classical Action in Conformal Gauge:}
\[S = \frac{1}{2\pi} \int d^2x \sqrt{-g} \left( \eta \left(R + \frac{8}{\ell^2} \right) - \frac{\ell^2}{4} F_{\mu\nu}F^{\mu\nu} \right)\]

\textbf{2. Twisted Stress Tensor:}  
Conformal diffeomorphisms do not preserve boundary conditions → must be accompanied by a \( U(1) \) gauge transformation.  
This leads to a \emph{twisted} energy-momentum tensor:
\[\widetilde{T}_{\pm\pm} = T_{\pm\pm} \pm \frac{E \ell^2}{4} \partial_\pm j_\pm\]

\textbf{3. Commutator of Twisted Stress Tensor:}
\[\widetilde{T}_{--}(x), \widetilde{T}_{--}(y)] = \cdots + \frac{\pi k E^2 \ell^4}{8} \partial_y^3 \delta(x - y)\]


\textbf{4. Central Charge:}
From the above commutator, extract the central term:
\[c = \frac{3k E^2 \ell^4}{4}\]

\textbf{Conclusion:}  
Twisting by a conserved \( U(1) \) current yields a Virasoro algebra with nonzero central charge, despite pure \(AdS_2\) gravity having \( c = 0 \).
\end{frame}

\begin{frame}[allowframebreaks]{Central Charge for JT Gravity (\cite{jackiw1985lower},\cite{teitelboim1983gravitation}) coupled to ModMax {\color{red}\scriptsize [HR,DRC]\cite{rathi2023ads2}}}

The authors in this paper started with the \textit{ModMax} lagrangian, which is an example of non-linear electrodynamics with one free parameter, $\beta$, so formed to possess the usual $\mathbb{SO}(2)$ symmetry of Maxwell's theory along with the conformal symmetry. Further in the weak field limit $\left( \beta \to \infty \right) $, it must yield the Maxwell theory. The \textit{ModMax} lagrangian is given by
\begin{equation}
    \label{eqn:modmax-lagrangian}
    \mathcal{L} = \frac{1}{2} \left( S \cosh\beta - \sqrt{S^2+P^2}\sinh\beta  \right)
\end{equation}

where $S=\frac{1}{2}F_{\mu\nu}F^{\mu\nu}$ and $P=\frac{1}{2}F_{\mu\nu}\tilde{F}^{\mu\nu}$. $\tilde{F}^{\mu\nu}$ is the hodge dual of the electromagnetic field tensor defined as $\tilde{F}^{\mu\nu} = \frac{1}{4}\epsilon^{\mu\nu\rho\sigma}F_{\rho\sigma}$.


\subsection{The bulk action and the equations of motion}
The 4D action for the gravity coupled to the \textit{ModMax} lagrangian is given as 
\begin{equation}
    I = \frac{1}{16\pi G_4}\int d^4x\sqrt{-g}\left(R - 2\Lambda - 4\alpha\mathcal{L}_{MM}\right)
\end{equation}

where $\alpha$ is the coupling constant and $\Lambda$ is the cosmological constant. The authors performed suitable dimensional reduction to obtain action for a $\left( 1+1 \right)$D \textit{JT} gravity theory. The ansatz for the metric was taken to be 
\begin{align}
    ds_{(3+1)}^2 &= g_{\mu\nu}(x^\rho)dx^\mu dx^\nu + \Phi(x^\mu)dx_i^2 \\
    A_\mu &\equiv A_\mu(x^\nu), \quad A_z \equiv A_z(x^\mu),
\end{align}

Here $\mu, \nu$ are the indices for coordinates in the reduced dimensions and $i$ denotes the compact dimensions. The $2D$ projected \textit{ModMax} action is written as 
\begin{align}
    I_2 &= \frac{1}{16\pi G_2}\int d^2x\sqrt{-g_{(2)}}\left(\Phi R^{(2)} - 2\Lambda\Phi - 4\alpha\Phi\mathcal{L}_{(M)}^{(2)}\right) \\ 
    \mathcal{L}_{\text{MM}}^{(2)} &= \frac{1}{2}\left(s\cosh\beta - \sqrt{s^2 + p^2}\sinh\beta\right) \\
    s &= \frac{1}{2}F_{\mu\nu}F^{\mu\nu} + \Phi^{-1}\left((\partial\chi)^2 + (\partial\zeta)^2\right), \quad p = -2\Phi^{-1}\epsilon^{\mu\nu}\partial_\mu\chi\partial_\nu\zeta
\end{align}

\subsection{Calculation of the central charge}
An approach similar to \cite{Castro:2008ne} was employed here, wherein the gauge transformation were obtained by imposing boundary condition preservation upon diffeomorphisms. And thus the coefficient of the modification term in the stress-energy tensor was identified as the central charge. The central charge thus obtained is given as
\begin{equation}
c_{3D} = \frac{1}{144\sqrt{3}\pi G_2}\left(\alpha - 12\beta\alpha + 2\alpha^2\right),
\end{equation}

The authors thus rightly concluded that in the limiting case of weak field $\left( \beta\to 0 \right) $ the central charge asymptotes to $\displaystyle\frac{1}{G_2}$ which matches with the central charge obtained by \cite{Castro:2008ne} for the case of Maxwell theory.



\end{frame}


