% Chapter Template

\chapter{Introduction} % Main chapter title

\label{Chapter1}

%\lhead{Chapter 1. \emph{Introduction}} % Change X to a consecutive number; this is for the header on each page - perhaps a shortened title
\lhead{\emph{Introduction}} % Change X to a consecutive number; this is for the header on each page - perhaps a shortened title

%----------------------------------------------------------------------------------------
%	SECTION 1
%----------------------------------------------------------------------------------------

What am I going to do in the thesis?

\section{Background}

The background for a theory of \textit{Quantum Gravity} is set by a very famous principle, known as the \textit{Holographic Principle}. $AdS_n-CFT_{n-1}$ duality is the prime example of this principle. 

The complexity of the mathematical models that we have to deal with increases exponentially with the dimensionality of space-time. So it is fruitful to study the consequences of these principles in lower dimensions. Thus it is customary to explore the $(1+1)D$ (known as the \textit{Jachiw-Teitelboim} gravity) or $(2+1)D$ theories of gravity.  


\subsection{Holographic Principle}

The Holographic Principle: A Scientific Explanation
The Holographic Principle is a revolutionary conjecture in theoretical physics that suggests the information content of a volume of space can be fully described by a theory defined on its boundary, much like a hologram encodes a three-dimensional image on a two-dimensional surface. Proposed in the 1990s by Gerard ’t Hooft and Leonard Susskind, it emerged from studies of black hole thermodynamics and quantum gravity, challenging our intuitive understanding of space, information, and the fundamental nature of the universe. The principle posits that the degrees of freedom within a region of space are not proportional to its volume (as one might expect in a three-dimensional world) but to the area of its boundary, implying that our seemingly three-dimensional reality might be a "projection" of a lower-dimensional system. This idea has profound implications for reconciling quantum mechanics with general relativity, particularly in the context of black holes and string theory, and it underpins modern approaches to quantum gravity like the AdS/CFT correspondence.
The Holographic Principle originated from the study of black holes, specifically the paradox of information loss. In the 1970s, Stephen Hawking showed that black holes emit radiation (now called Hawking radiation) due to quantum effects near the event horizon, leading to their eventual evaporation. This process suggested that information inside the black hole might be lost, violating quantum mechanics’ principle of unitarity, which demands that information is preserved. Jacob Bekenstein’s work on black hole entropy provided a crucial clue: the entropy ( S ) of a black hole is proportional to the area ( A ) of its event horizon, not its volume. This is encapsulated in the Bekenstein-Hawking entropy formula:
$$ S = \frac{k c^3 A}{4 \hbar G}, $$
where \( k \) is Boltzmann’s constant, \( c \) is the speed of light, \( \hbar \) is the reduced Planck constant, \( G \) is the gravitational constant, and \( A = 4\pi r^2 \) for a Schwarzschild black hole with radius \( r = \frac{2GM}{c^2} \). Since entropy measures the number of microstates (or information content) of a system, this formula implies that the information inside a black hole scales with the two-dimensional area of its horizon (in Planck units, \( A/4l_p^2 \), where \( l_p = \sqrt{\frac{\hbar G}{c^3}} \) is the Planck's length), not its three-dimensional volume. This was surprising because, in most physical systems, entropy scales with volume, reflecting the number of particles or states within.
’t Hooft and Susskind generalized this observation, proposing that the information content of any region of space, not just black holes, is limited by the area of its boundary. Specifically, the maximum entropy in a region enclosed by a surface of area ( A ) is given by the Bekenstein bound:
$$ S \leq \frac{k c^3 A}{4 \hbar G}. $$
This bound suggests that the number of quantum states (or bits of information) needed to describe everything inside a volume is encoded on its boundary, with roughly one bit per Planck area (\( l_p^2 \)). For example, a spherical region of radius \( r \) has a boundary area \( A = 4\pi r^2 \), so its maximum entropy is proportional to \( r^2 \), not \( r^3 \). This is the essence of the Holographic Principle: the physics of a \( d \)-dimensional volume can be described by a theory in \( d-1 \) dimensions on its boundary, much like a hologram projects a 3D image from a 2D film.
The principle gained traction with the development of the AdS/CFT correspondence, proposed by Juan Maldacena in 1997, which provides a concrete realization of holography. AdS/CFT conjectures a duality between a gravitational theory in ( d )-dimensional Anti-de Sitter (AdS) space (a universe with a negative cosmological constant) and a conformal field theory (CFT) on its ( (d-1) )-dimensional boundary. For instance, in the most studied case, type IIB string theory in \( AdS_5 \times S^5 \) (a five-dimensional AdS space times a five-dimensional sphere) is dual to a four-dimensional \( \mathcal{N}=4 \) supersymmetric Yang-Mills theory on the boundary. The CFT, a quantum field theory without gravity, fully encodes the dynamics of the AdS bulk, including gravity, black holes, and quantum effects. The duality implies that bulk phenomena, like the formation of a black hole, correspond to specific states or operators in the boundary CFT. Mathematically, the partition functions of the two theories are equal:
$$ Z_{\text{AdS}} = Z_{\text{CFT}}, $$
where \( Z = \text{Tr}(e^{-\beta H}) \) is the partition function, \( H \) is the Hamiltonian, and \( \beta = 1/(k T) \). This equivalence allows physicists to study complex gravitational phenomena, like quantum gravity, using well-understood quantum field theories.
The Holographic Principle has far-reaching implications. First, it suggests that gravity, traditionally described by general relativity in the bulk, may be an emergent phenomenon arising from quantum interactions on the boundary. In AdS/CFT, the metric of the AdS space is encoded in the CFT’s correlation functions, and the radial dimension of AdS corresponds to the energy scale in the CFT via the renormalization group flow. This is often expressed through the Ryu-Takayanagi formula, which relates the entanglement entropy \( S_{\text{EE}} \) of a region in the CFT to the area of a minimal surface in the AdS bulk:
$$ S_{\text{EE}} = \frac{\text{Area of minimal surface}}{4 G \hbar}. $$
This formula generalizes the Bekenstein-Hawking entropy to arbitrary regions and highlights the deep connection between quantum entanglement and geometry.
Second, the principle challenges our understanding of spacetime. If a 3D universe can be described by a 2D boundary, the extra dimension may be an illusion, much like a hologram creates the appearance of depth. This raises questions about the fundamental nature of reality: is our universe holographic, with physical laws emerging from a lower-dimensional theory? While AdS/CFT applies to AdS spaces, efforts are underway to extend holography to flat spacetimes (like our universe) or de Sitter spaces, though these are less understood.
Third, the Holographic Principle constrains quantum gravity theories. Any consistent theory must respect the area scaling of entropy, ruling out models where information scales with volume. This has influenced string theory, loop quantum gravity, and other approaches, pushing physicists to rethink locality and causality.
Despite its elegance, the Holographic Principle faces challenges. Outside AdS/CFT, explicit holographic dualities for realistic spacetimes are lacking. The principle also raises philosophical questions: if reality is a hologram, what is the "true" dimensionality of the universe? Moreover, encoding bulk dynamics on a boundary requires non-local interactions, which are hard to reconcile with local field theories.
In conclusion, the Holographic Principle is a cornerstone of modern theoretical physics, bridging black hole thermodynamics, quantum mechanics, and gravity. By suggesting that the universe’s information is encoded on a lower-dimensional boundary, it offers a path to unify quantum mechanics and general relativity. Equations like the Bekenstein-Hawking entropy, Bekenstein bound, and Ryu-Takayanagi formula quantify this idea, while AdS/CFT provides a concrete framework. As research progresses, the principle may unlock deeper truths about the universe, perhaps revealing that reality, like a hologram, is a projection of a more fundamental truth.


\subsection{AdS$_n$-CFT$_{n-1}$ Duality}

\subsubsection{AdS$_n$ spacetime}


\subsubsection{CFT$_{n-1}$}

\subsubsection{The duality}

AdS/CFT Duality Explanation
The AdS/CFT correspondence, a cornerstone of modern theoretical physics, posits a duality between a conformal field theory (CFT) in ( d )-dimensional Minkowski space and a gravitational theory in ( (d+1) )-dimensional Anti-de Sitter (AdS) space. Introduced by Maldacena, this holographic principle suggests that a strongly coupled CFT, such as \( \mathcal{N}=4 \) super Yang-Mills (SYM) in four dimensions, is equivalent to a weakly coupled supergravity theory in \( AdS_5 \times S^5 \). The correspondence leverages the isometry group \( SO(2,4) \) of \( AdS_5 \), which matches the conformal group of the CFT, enabling a dictionary between CFT operators and AdS fields. For instance, a CFT operator \( O(x) \) with conformal dimension \( \Delta \) couples to a bulk field \( \phi(x, u) \) via boundary interactions, described by the action term \( \int d^4 x , \phi_0(x) O(x) \), where \( \phi_0(x) \) is the boundary value of \( \phi \). Correlation functions in the CFT are computed using the bulk partition function, approximated classically as \[Z_{\text{AdS}}[\phi_0] = e^{-S_{\text{AdS}}[\phi]} \approx \langle e^{\int \phi_0 O} \rangle_{\text{CFT}}\] The large ( N ) limit of the gauge theory, with ’t Hooft coupling \( \lambda = g_{\text{YM}}^2 N \), corresponds to the classical supergravity limit when \( \lambda \gg 1 \), while the string coupling \( g_s \sim 1/N \). This duality allows quantum effects in the CFT, like correlation functions \[\langle O(x_1) \cdots O(x_n) \rangle\] to be computed via classical gravitational dynamics, providing insights into strongly coupled systems, confinement, and even condensed matter physics, by mapping complex quantum phenomena to tractable geometric problems in AdS space.

\subsection{$AdS_2$-CFT$_1$ correspondence}

\subsubsection{2D gravity models - JT gravity}
Jackiw-Teitelboim (JT) gravity is a model of two-dimensional (2D) gravity that provides a valuable framework for understanding quantum gravity and black hole thermodynamics in a simplified setting. Unlike in four dimensions where Einstein's equations produce dynamic degrees of freedom for the metric, in 2D spacetime the Einstein tensor vanishes identically due to topological constraints. Specifically, for any 2D metric $g_{\mu\nu}$, the Einstein tensor $G_{\mu\nu} = R_{\mu\nu} - \frac{1}{2}g_{\mu\nu}R$ is identically zero because the Ricci tensor $R_{\mu\nu}$ is completely determined by the scalar curvature $R$, and the variation of the Einstein-Hilbert action yields no dynamics. Therefore, to have a nontrivial theory of gravity in 2D, auxiliary fields such as the *dilaton* $\phi$ are introduced.

In JT gravity, the action is constructed as

$$
S_{\text{JT}} = \int d^2x \sqrt{-g} \, \phi (R + 2\Lambda),
$$

where $\phi$ is the dilaton field, $R$ is the Ricci scalar of the 2D spacetime metric $g_{\mu\nu}$, and $\Lambda$ is a cosmological constant (often negative, corresponding to an AdS$_2$ background). This action yields second-order field equations for the metric and first-order equations for the dilaton, making the theory solvable. Varying the action with respect to $\phi$ gives

$$
R + 2\Lambda = 0,
$$

which fixes the geometry to a constant curvature spacetime—typically anti-de Sitter (AdS$_2$) for $\Lambda < 0$. Varying the action with respect to the metric gives another equation involving derivatives of $\phi$, which governs how the dilaton profiles across the spacetime.

A key feature of JT gravity is that, despite having no propagating degrees of freedom in the metric, it supports black hole solutions, and its boundary dynamics are nontrivial. These dynamics are governed by the Schwarzian action, which appears when considering the low-energy limit of the boundary mode:

$$
S_{\text{Sch}}[f] = -C \int dt \, \{ f(t), t \},
$$

where $\{f(t), t\}$ is the Schwarzian derivative of the boundary reparametrization $f(t)$, and $C \propto \phi_r$ is related to the value of the dilaton at the boundary. This boundary action captures the low-energy dynamics of near-extremal black holes in higher-dimensional theories and connects JT gravity to the SYK model—a disordered quantum mechanical system with similar infrared behavior.

The JT model also arises from dimensional reduction of higher-dimensional gravity theories. For instance, spherically reducing four-dimensional Einstein gravity under the ansatz

$$
ds^2_{(4)} = g_{\mu\nu}(x) dx^\mu dx^\nu + \Phi^2(x) d\Omega_2^2
$$

produces an effective 2D theory for $g_{\mu\nu}$ and the scalar $\Phi$, leading to actions of the form

$$
S = \int d^2x \sqrt{-g} \left[ \Phi^2 R - \frac{1}{2} (\nabla \Phi)^2 - V(\Phi) \right],
$$

which includes the JT gravity model in specific limits where $\Phi \sim \phi$, and kinetic terms may be neglected.

The beauty of JT gravity lies in its exact solvability and the ability to study non-perturbative aspects of quantum gravity, black hole entropy, and holography in a tractable setting. The model encapsulates the essence of diffeomorphism invariance, black hole thermodynamics, and quantum effects such as Hawking radiation, despite its apparent simplicity.


\subsubsection{1D CFT model - SYK model}
The Sachdev-Ye-Kitaev (SYK) model is a quantum mechanical system of $N$ Majorana fermions $\chi_i(\tau)$ with all-to-all random $q$-body interactions, most commonly $q = 4$. Its action is given by:

$$
S = \int d\tau \left[ \frac{1}{2} \sum_{i=1}^{N} \chi_i \partial_\tau \chi_i - \frac{1}{4!} \sum_{i,j,k,l} J_{ijkl} \chi_i \chi_j \chi_k \chi_l \right],
$$

where $J_{ijkl}$ are real, antisymmetric, Gaussian-random couplings with zero mean and variance

$$
\langle J_{ijkl}^2 \rangle = \frac{3! J^2}{N^3}.
$$

At large $N$, the dominant Feynman diagrams contributing to the two-point function are “melon” diagrams, which can be resummed via a Schwinger-Dyson equation involving the full two-point function $G(\tau) = \frac{1}{N} \sum_i \langle \chi_i(\tau) \chi_i(0) \rangle$ and its self-energy $\Sigma(\tau)$:

$$
\Sigma(\tau) = J^2 G(\tau)^3, \qquad G(i\omega)^{-1} = -i\omega - \Sigma(i\omega).
$$

In the infrared (IR) limit, where $J|\tau| \gg 1$, the kinetic term $\partial_\tau$ becomes negligible and the equations become conformally invariant. The solution to the Schwinger-Dyson equations in this regime is:

$$
G(\tau) = b \frac{\text{sgn}(\tau)}{|J \tau|^{2\Delta}}, \quad \Delta = \frac{1}{4}, \quad b^4 = \frac{1}{4\pi}.
$$

This emergent conformal symmetry is spontaneously and explicitly broken, leading to the appearance of a soft mode governed by the Schwarzian action. By considering reparametrizations $\tau \to f(\tau)$, one finds that the leading IR effective action for these modes is:

$$
S_{\text{Sch}} = -\alpha_S N \int d\tau \, \text{Sch}(f(\tau), \tau), \quad \text{where} \quad \text{Sch}(f(\tau), \tau) = \frac{f'''(\tau)}{f'(\tau)} - \frac{3}{2} \left( \frac{f''(\tau)}{f'(\tau)} \right)^2.
$$

This Schwarzian action governs the breaking of reparametrization symmetry from $\text{Diff}(R)$ to $SL(2, \mathbb{R})$, and it plays a central role in the model's connection to two-dimensional dilaton gravity and holography.

The four-point function is given by a sum over ladder diagrams built from full propagators and a kernel:

$$
K(\tau_1, \tau_2; \tau_3, \tau_4) = -J^2 (q-1) G(\tau_{13}) G(\tau_{24}) G(\tau_{34})^{q-2}.
$$

This kernel acts on bilocal functions, and its eigenfunctions are the conformal partial waves $\mathcal{F}_h(\tau_i)$ of the $SL(2, \mathbb{R})$ group. The resulting four-point function in the conformal limit takes the form:

$$
\langle \chi_i(\tau_1) \chi_i(\tau_2) \chi_j(\tau_3) \chi_j(\tau_4) \rangle = G(\tau_{12})G(\tau_{34}) + \frac{1}{N} \int \frac{dh}{2\pi i} \rho(h) \mathcal{F}_h(\tau_i),
$$

with

$$
\rho(h) = \mu(h) \frac{k(h)}{1 - k(h)},
$$

where $k(h)$ are the eigenvalues of the kernel and $\mu(h)$ is a measure factor involving gamma functions. The poles of $\rho(h)$, given by $k(h) = 1$, determine the dimensions $h$ of bilinear operators exchanged in the four-point function, such as $\mathcal{O}_h = \sum_i \chi_i \partial_\tau^{2n+1} \chi_i$.

Of particular importance is the $h = 2$ mode, which corresponds to the Schwarzian sector. It leads to a divergence in the conformal four-point function and signals the need to include the full effective action, incorporating non-conformal corrections. The dominance of the $h = 2$ exchange in out-of-time-ordered correlators implies maximal quantum chaos, with a Lyapunov exponent:

$$
\lambda_L = \frac{2\pi}{\beta},
$$

which saturates the bound on chaos in quantum systems and matches the behavior of black holes in Einstein gravity. This profound connection makes the SYK model a valuable tool in exploring the AdS/CFT correspondence and quantum aspects of gravity.

\subsection{Central Charge}


\section{Motivation}

{\color{red} Central charge calculation ...}

%\section{Organization of the Thesis}
%
%This is how acronym is added \ac{EBSD}. The present dissertation is divided into eight chapters, each of which is further divided into well-structured sections and subsections. \autoref{Chapter1} explains the rationale behind the dissertation and its objectives. \autoref{Chapter2} says something more. 
%
