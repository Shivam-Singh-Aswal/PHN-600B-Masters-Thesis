\documentclass[11pt, a4paper, twoside]{Thesis} % Paper size, default font size and one-sided paper
\usepackage{floatrow}
\floatsetup[table]{capposition=top}
\usepackage{textcomp}
\usepackage{wrapfig}
\usepackage{lscape}
\usepackage{rotating}
\usepackage{graphicx}
\usepackage{caption}
\usepackage{amsmath}
\usepackage{amssymb}
\usepackage{upgreek}
\usepackage{gensymb}
\usepackage{csquotes}
\usepackage{pdfpages}
\usepackage{lipsum}
\usepackage{notoccite}
\renewcommand{\chapterautorefname}{Chapter} % Forces capital "Chapter"
%\usepackage[open]{bookmark}

% acronyms
\usepackage{acronym}
\usepackage{array}
\newcolumntype{L}{>{\centering\arraybackslash}m{3cm}}

%\usepackage[acronym]{glossaries}
% prints author names as small caps


\makeatletter
\AtBeginDocument{%
  \renewcommand*{\AC@hyperlink}[2]{%
    \begingroup
      \hypersetup{hidelinks}%
      \hyperlink{#1}{#2}%
    \endgroup
  }%
}
\makeatother





%\usepackage{subcaption} %incompatible with subfig
\graphicspath{{Pictures/}} % Specifies the directory where pictures are stored
\usepackage[square, numbers]{natbib} % Use the natbib reference package - read up on this to edit the reference style; if you want text (e.g. Smith et al., 2012) for the in-text references (instead of numbers), remove 'numbers' v

\hypersetup{urlcolor=black, colorlinks=true} % Colors hyperlinks in blue - change to black if annoyingv`
\title{\ttitle} % Defines the thesis title - don't touch this

\begin{document}
\makeatletter
\renewcommand*{\NAT@nmfmt}[1]{\textsc{#1}}
\makeatother

% prints author names as small caps


\frontmatter % Use roman page numbering style (i, ii, iii, iv...) for the pre-content pages

\setstretch{1.5} % Line spacing of 1.6 (double line spacing)

% Define the page headers using the FancyHdr package and set up for one-sided printing
\fancyhead{} % Clears all page headers and footers
\rhead{\thepage} % Sets the right side header to show the page number
\lhead{} % Clears the left side page header

\pagestyle{fancy} % Finally, use the "fancy" page style to implement the FancyHdr headers

\newcommand{\HRule}{\rule{\linewidth}{0.5mm}} % New command to make the lines in the title page

% PDF meta-data
\hypersetup{pdftitle={\ttitle}}
\hypersetup{pdfsubject=\subjectname}
\hypersetup{pdfauthor=\authornames}
\hypersetup{pdfkeywords=\keywordnames}

%----------------------------------------------------------------------------------------
%	TITLE PAGE
%----------------------------------------------------------------------------------------

\begin{titlepage}
\begin{center}

\HRule \\[0.4cm] % Horizontal line
{\huge \bfseries \ttitle}\\[0.4cm] % Thesis title
\HRule \\[1cm] % Horizontal line

\graphicspath{ {./Figures/} }
\begin{figure}[hb]
  \centering
  %\includegraphics[width=0.35\linewidth]{Pictures/iitk_logo.png}
  \includegraphics[width=0.35\linewidth]{Pictures/logo_color.jpg}
\end{figure}



\large \textit{A thesis submitted in fulfillment of the requirements\\ for the degree of \degreename}\\[0.3cm] % University requirement text
\textit{by}\\[0.4cm]
\textbf{\authornames}\\
Enrollment: \textit{20311021}\\[0.4cm] % Author name

\textbf{Supervisor: }\supnameA\\



\vfill
\vfill 

\DEPTNAME\\ % Research group name and department name
\textsc{ \UNIVNAME}\\[1.5cm] % University name
\large \today\\[2cm] % Date


\end{center}

\end{titlepage}


%----------------------------------------------------------------------------------------
%   Copyright Page
%----------------------------------------------------------------------------------------

\thispagestyle{empty} % Page style needs to be empty for this page
\hspace{0pt}
\vfill 
\begin{center}
    \textbf{\copyright \UNIVNAME, 2025\\ALL RIGHTS RESERVED}
\end{center}
\vfill 
\hspace{0pt}

\clearpage 


%----------------------------------------------------------------------------------------
%	DECLARATION PAGE
%	Your institution may give you a different text to place here
%----------------------------------------------------------------------------------------

\StudentDeclaration{\addtocontents{toc}{}} % Add a gap in the Contents, for aesthetics

%This is to certify that the thesis titled \textbf{``\ttitle''} has been authored by me. It presents the research conducted by me under the supervision of \textbf{\supnameA} and \textbf{\supnameB}.\par
This is to certify that the thesis titled \textbf{``\ttitle''} has been authored by me. It presents the research conducted by me under the supervision of \textbf{\supnameA} in partial fulfillment of the requirements for the award of the degree of \textbf{\degreename} and submitted in the \textbf{\deptname} of the \textbf{\univname}.\par

To the best of my knowledge, it is an original work, both in terms of research content and narrative, and has not been submitted elsewhere, in part or in full, for a degree.\\ [2cm]
\begin{minipage}{\textwidth}
		\begin{flushright}
            {\textbf{\authornames}\\ \textbf{Enrollment No. 20311021}\\
                \normalsize{\href{https://ph.iitr.ac.in/}{\textbf{\deptname}}\\
            \textbf{\univname}}}
		\end{flushright}
\end{minipage}

\vfill
This is to certify that the above statement made by the candidate is correct to the best of my knowledge.\\[1cm]

\begin{minipage}{0.39\textwidth}
		\begin{flushleft}
            {\textbf{Date:}}
		\end{flushleft}
\end{minipage}
\begin{minipage}{0.59\textwidth}
		\begin{flushright}
            {\textbf{Supervisor\\(\supnameA})} 
		\end{flushright}
\end{minipage}


\vfill

\clearpage % Start a new page

\Declaration{\addtocontents{toc}{}} % Add a gap in the Contents, for aesthetics
%\setcounter{page}{1}

\begin{minipage}{\textwidth}
    
    It is certified that the work contained in this thesis entitled \textbf{\enquote{\ttitle}} by \textbf{\authornames} in partial fulfillment of the requirements for the award of the degree of \textbf{\degreename} and submitted in the \textbf{\deptname} of the \textbf{\univname} has been carried out under my supervision and is an authentic record of his own work carried out during the period from July 2024 to May 2025 and that it has not been submitted elsewhere for a degree.
        
\end{minipage}

\vspace{20.00mm}

\begin{minipage}{0.39\textwidth}
		\begin{flushleft}
            {\textbf{Date:}}
		\end{flushleft}
\end{minipage}
\begin{minipage}{0.59\textwidth}
		\begin{flushright}
            {\textbf{\supnameA}\\ 
            \textbf{Associate Professor}\\
            \textbf{\deptname}\\
            \textbf{\univname}}
		\end{flushright}
\end{minipage}
%\begin{minipage}{0.49\textwidth}
%	\begin{flushleft} \large
%			{\supnameB\\
%			Associate Professor\\
%			\deptname\\
%			\univname}
%	\end{flushleft}		
%\end{minipage}
\vfill
\

\clearpage % Start a new page
%----------------------------------------------------------------------------------------
%	ABSTRACT PAGE
%----------------------------------------------------------------------------------------

\addtotoc{Abstract} % Add the "Abstract" page entry to the Contents
\lhead{\emph{Abstract}}

\abstract{\addtocontents{toc}{} % Add a gap in the Contents, for aesthetics
%\abstract{\addtocontents{toc}{\vspace{0.5em}} % Add a gap in the Contents, for aesthetics

%\lipsum[1]

This thesis investigates lower-dimensional theories of gravity and their dual conformal field theories within the framework of the AdS/CFT correspondence and the Holographic Principle. Emphasis is placed on two- and three-dimensional models where gravity is coupled to ModMax electrodynamics --- a non-linear generalization of Maxwell theory that retains conformal symmetry and the SO(2) electric-magnetic duality. The central theme of the thesis, particularly in Chapter 3, is the attempt to calculation and comparison of central charges in these lower-dimensional theories, which play a crucial role in characterizing the dual boundary CFTs.

We begin with a four-dimensional Einstein-Hilbert action coupled to the ModMax Lagrangian. The theory is then dimensionally reduced to obtain effective three-dimensional and two-dimensional gravitational actions. This reduction is carried out using consistent metric ansatz, which preserve the conformal structure and allow us to derive equations of motion, stress-energy tensors, and gauge field dynamics in the lower-dimensional frameworks.

In the three-dimensional case, the central charge is computed by analyzing the transformation of the boundary stress-energy tensor under diffeomorphisms and accompanying gauge transformations that preserve the boundary conditions. 

%The coefficient of the anomalous term in the stress tensor transformation gives the central charge:
\[
%c_{3D} = \frac{1}{144\sqrt{3}\pi G_2}(\alpha - 12\beta\alpha + 2\alpha^2),
\]
%where $\alpha$ is a coupling constant and $\beta$ is the ModMax non-linearity parameter. This expression generalizes the result for the Maxwell case, which is recovered in the $\beta \to 0$ limit.

Subsequently, a further reduction from three to two dimensions is to be performed to derive the 2D effective action. This process also involves an appropriate choice of boundary conditions and the use of the Fefferman-Graham gauge to facilitate asymptotic analysis. The 2D central charge is then extracted using a similar method, analyzing the transformation of the boundary stress-energy tensor under combined conformal and gauge symmetries.

In conclusion, in this work we set out to demonstrate that central charge calculations in ModMax-coupled gravity are tractable in both 2D and 3D settings. However the equations of motions were too complex to be solved in the limiited time we had. Thus it will be taken as a future endeavour to carry out the calculation with the choices of metric made. }





%----------------------------------------------------------------------------------------
%	Declaration Page
%----------------------------------------------------------------------------------------



%----------------------------------------------------------------------------------------
%	ACKNOWLEDGEMENTS
%----------------------------------------------------------------------------------------
\clearpage % Start a new page
\setstretch{1.3} % Reset the line-spacing to 1.3 for body text (if it has changed)
\lhead{\emph{Acknowledgements}}
\acknowledgements{\addtocontents{toc}{} % Add a gap in the Contents, for aesthetics

%I would like to express my sincere gratitude to my supervisor, \supnameA, for his invaluable guidance and support throughout my research journey. His expertise and insights have been instrumental in shaping my understanding of the subject matter and have greatly contributed to the success of this thesis. 
%
%This thesis work marks the end of my journey as an undergraduate Physics student, during which I have had the privilege of learning from many exceptional professors. I would like to extend my heartfelt thanks to all my professors for their unwavering support and encouragement, especially Professor Aalok Misra, who has been a source of constant support and inspiration. I would like to express my gratitude to my friends with whom I had innumerable discussions
%on Physics, Math and life in general. 

I would like to express my sincere gratitude to my supervisor, \supnameA, for his invaluable guidance and support throughout my research journey. His expertise and insights have been instrumental in shaping my understanding of the subject matter and have greatly contributed to the success of this thesis.

This thesis work marks the end of my journey as an undergraduate Physics student, during which I have had the privilege of learning from many exceptional professors. I would like to extend my heartfelt thanks to all my professors for their unwavering support and encouragement, especially Professor Aalok Misra, who has been a source of constant support and inspiration.

I would like to express my gratitude to my friends with whom I had innumerable discussions on Physics, Math, and life in general. Their companionship and perspectives enriched my academic experience and made the journey all the more enjoyable.

I am deeply grateful to my parents for their unconditional love, patience, and encouragement throughout my academic journey. Their belief in me has been a constant source of strength and motivation, especially during challenging times. Without their support, this thesis and the completion of my degree would not have been possible.

Finally, I would like to thank everyone who, in one way or another, contributed to my personal and academic growth over the past few years. Each interaction, challenge, and success has helped shape who I am today, and for that, I am truly thankful.

}
\clearpage % Start a new page

%----------------------------------------------------------------------------------------
%	LIST OF CONTENTS/FIGURES/TABLES PAGES
%----------------------------------------------------------------------------------------

\pagestyle{fancy} % The page style headers have been "empty" all this time, now use the "fancy" headers as defined before to bring them back

\lhead{\emph{Contents}} % Set the left side page header to "Contents"
\tableofcontents % Write out the Table of Contents

%\lhead{\emph{List of Figures}} % Set the left side page header to "List of Figures"
%\listoffigures % Write out the List of Figures
%
%\lhead{\emph{List of Tables}} % Set the left side page header to "List of Tables"
%\listoftables % Write out the List of Tables

%----------------------------------------------------------------------------------------
%	ABBREVIATIONS
%----------------------------------------------------------------------------------------




%\clearpage % Start a new page
%
%\setstretch{1.5} % Set the line spacing to 1.5, this makes the following tables easier to read
%
%\lhead{\emph{Abbreviations}} % Set the left side page header to "Abbreviations"
%
%\chapter*{Abbreviations}
%\addtotoc{Abbreviations}
%\begin{acronym}[XXXXXXXXX] % Give the longest label here so that the list is nicely aligned
\acro{EBSD}{electron backscatter diffraction}
% \acro{SEM}{scanning electron microscope}
% \acro{WHA}{tungsten heavy alloys}
% \acro{SPS}{spark plasma sintering}
% \acro{ND}{normal direction to the rolling plane}
% \acro{CD}{compression direction}
% \acro{RD}{rolling direction in a rolled plate}
% \acro{LD}{Loading direction}
% \acro{TD}{transverse direction in a rolled plate}
% \acro{XRD}{X-ray diffraction}
% \acro{ODF}{orientation distribution function}
% \acro{OM}{optical microscope}
% \acro{IQ}{image quality}
% \acro{IPF}{inverse pole figure}
% \acro{FESEM}{field emission scanning electron microscopy}
% \acro{f.c.c.}{face centred cubic}
% \acro{b.c.c.}{body centred cubic}
% \acro{EPMA}{electron probe microanalyser}
% \acro{EDS}{energy dispersive spectrometer}
% \acro{LPS}{liquid phase sintering}
% \acro{FFT}{fast Fourier transform}
% \acro{HEM}{homogeneous effective medium}
% \acro{RVE}{representative volume element}
% \acro{CRSS}{critical resolved shear stress}
% \acro{PF}{pole figure}
% \acro{VPSC}{viscoplastic self-consistent}
% \acro{STECA}{Stress Texture Calculator}
% \acro{DAMASK}{D\"{u}sseldorf Advanced Material Simulation Kit}
% \acro{KAM}{kernal average misorientation}
% \acro{GND}{geometry necessary dislocation}
% \acro{GROD}{grain reference orientation deviation}
% \acro{CAD}{computer-aided design}
% \acro{AVACS}{average number of active systems per grain}
% \acro{SRO}{short range ordering}
% \acro{TWIP}{twinning induced plasticity}
% \acro{TRIP}{transformation induced plasticity}
% \acro{h.c.p.}{hexagonal close packed}
% \acro{YS}{yield strength}
% \acro{UTS}{ultimate tensile strength}
% \acro{MSBs}{micro-shear bands}
% \acro{DB}{deformation bands}
% \acro{TB}{transition bands}
% \acro{SFE}{stacking fault energy}
% \acro{SHPB}{split-Hopkinson pressure bar}
% \acro{ppm}{parts per million}
% \acro{CES}{Cambridge Engineering Selector}
% \acro{DBTT}{ductile-to-brittle transition temperature}
% \acro{mrd}{multiples of a random distribution}
% \acro{FWHM}{full width at half maximum}
% \acro{CPFEM}{crystal plasticity finite element methods}

\end{acronym}




%----------------------------------------------------------------------------------------
%	PHYSICAL CONSTANTS/OTHER DEFINITIONS
%----------------------------------------------------------------------------------------
%
% \clearpage % Start a new page

% \lhead{\emph{Physical Constants}} % Set the left side page header to "Physical Constants"

% \listofconstants{lrcl} % Include a list of Physical Constants (a four column table)
% {
% Speed of Light & $c$ & $=$ & $2.997\ 924\ 58\times10^{8}\ \mbox{ms}^{-\mbox{s}}$ (exact)\\
% % Constant Name & Symbol & = & Constant Value (with units) \\
% }

%----------------------------------------------------------------------------------------
%	SYMBOLS
%----------------------------------------------------------------------------------------

%\clearpage % Start a new page
%
%\lhead{\emph{Symbols}} % Set the left side page header to "Symbols"
%
%\listofnomenclature{lll} % Include a list of Symbols (a two column table)
%{
%% Symbol & Name & Unit \\
%$C_g$ & contiguity \\
%$\phi_d$ & Dihedral angle
%}

%\ListofPublications{\addtocontents{toc}{\vspace{1em}} % Add a gap in the Contents, for aesthetics
%
%\textbf{Publications from Thesis}
%\begin{enumerate}
%
%    \item Paper 1  \href{https://doi.org/10.1016/j.ijrmhm.2022.105849}{\textit{\textcolor{blue}{10.1016/j.ijrmhm.2022.105849}}}.
%    
%    \item Paper 2. \href{https://doi.org/10.1016/j.matchar.2022.112010}{\textit{\textcolor{blue}{10.1016/j.matchar.2022.112010}}}.
%    
%    \item Paper 3. \href{https://doi.org/10.1007/s11661-021-06586-x}{\textit{\textcolor{blue}{10.1007/s11661-021-06586-x}}}.
%    \end{enumerate}
%    
%\textbf{Others}
%
%\begin{enumerate}
%    \item Paper 4 \href{https://doi.org/10.1080/02670836.2021.2007455}{\textit{\textcolor{blue}{10.1080/02670836.2021.2007455}}}.
%    
%    \item Paper 5 \href{https://doi.org/10.1080/02670836.2021.1946949}{\textit{\textcolor{blue}{10.1080/02670836.2021.1946949}}}.
%    
%    \item Paper 6 \href{https://doi.org/10.1080/02670836.2020.1773036}{\textit{\textcolor{blue}{10.1080/02670836.2020.1773036}}}.
%    
%    \item Paper 7 \href{https://doi.org/10.2139/ssrn.4125910}{\textit{\textcolor{blue}{10.2139/ssrn.4125910}}}.
%    
%\end{enumerate}
%}









% \\[2cm]


%\vfill{}


%\clearpage % Start a new page





\setstretch{1.3} % Return the line spacing back to 1.3
%
\pagestyle{empty} % Page style needs to be empty for this page
%
\dedicatory{Dedicated to my family.
} % Dedication text
%
\addtocontents{toc}{\vspace{1em}} % Add a gap in the Contents, for aesthetics

%----------------------------------------------------------------------------------------
%	THESIS CONTENT - CHAPTERS
%----------------------------------------------------------------------------------------

\mainmatter % Begin numeric (1,2,3...) page numbering

\pagestyle{fancy} % Return the page headers back to the "fancy" style

% Include the chapters of the thesis as separate files from the Chapters folder
% Uncomment the lines as you write the chapters

% Chapter Template

\chapter{Introduction} % Main chapter title

\label{Chapter1}

%\lhead{Chapter 1. \emph{Introduction}} % Change X to a consecutive number; this is for the header on each page - perhaps a shortened title
\lhead{\emph{Introduction}} % Change X to a consecutive number; this is for the header on each page - perhaps a shortened title

%----------------------------------------------------------------------------------------
%	SECTION 1
%----------------------------------------------------------------------------------------

What am I going to do in the thesis?

\section{Background}

The background for a theory of \textit{Quantum Gravity} is set by a very famous principle, known as the \textit{Holographic Principle}. $AdS_n-CFT_{n-1}$ duality is the prime example of this principle. 

The complexity of the mathematical models that we have to deal with increases exponentially with the dimensionality of space-time. So it is fruitful to study the consequences of these principles in lower dimensions. Thus it is customary to explore the $(1+1)D$ (known as the \textit{Jachiw-Teitelboim} gravity) or $(2+1)D$ theories of gravity.  


\subsection{Holographic Principle}

The Holographic Principle: A Scientific Explanation
The Holographic Principle is a revolutionary conjecture in theoretical physics that suggests the information content of a volume of space can be fully described by a theory defined on its boundary, much like a hologram encodes a three-dimensional image on a two-dimensional surface. Proposed in the 1990s by Gerard ’t Hooft and Leonard Susskind, it emerged from studies of black hole thermodynamics and quantum gravity, challenging our intuitive understanding of space, information, and the fundamental nature of the universe. The principle posits that the degrees of freedom within a region of space are not proportional to its volume (as one might expect in a three-dimensional world) but to the area of its boundary, implying that our seemingly three-dimensional reality might be a "projection" of a lower-dimensional system. This idea has profound implications for reconciling quantum mechanics with general relativity, particularly in the context of black holes and string theory, and it underpins modern approaches to quantum gravity like the AdS/CFT correspondence.
The Holographic Principle originated from the study of black holes, specifically the paradox of information loss. In the 1970s, Stephen Hawking showed that black holes emit radiation (now called Hawking radiation) due to quantum effects near the event horizon, leading to their eventual evaporation. This process suggested that information inside the black hole might be lost, violating quantum mechanics’ principle of unitarity, which demands that information is preserved. Jacob Bekenstein’s work on black hole entropy provided a crucial clue: the entropy ( S ) of a black hole is proportional to the area ( A ) of its event horizon, not its volume. This is encapsulated in the Bekenstein-Hawking entropy formula:
$$ S = \frac{k c^3 A}{4 \hbar G}, $$
where \( k \) is Boltzmann’s constant, \( c \) is the speed of light, \( \hbar \) is the reduced Planck constant, \( G \) is the gravitational constant, and \( A = 4\pi r^2 \) for a Schwarzschild black hole with radius \( r = \frac{2GM}{c^2} \). Since entropy measures the number of microstates (or information content) of a system, this formula implies that the information inside a black hole scales with the two-dimensional area of its horizon (in Planck units, \( A/4l_p^2 \), where \( l_p = \sqrt{\frac{\hbar G}{c^3}} \) is the Planck's length), not its three-dimensional volume. This was surprising because, in most physical systems, entropy scales with volume, reflecting the number of particles or states within.
’t Hooft and Susskind generalized this observation, proposing that the information content of any region of space, not just black holes, is limited by the area of its boundary. Specifically, the maximum entropy in a region enclosed by a surface of area ( A ) is given by the Bekenstein bound:
$$ S \leq \frac{k c^3 A}{4 \hbar G}. $$
This bound suggests that the number of quantum states (or bits of information) needed to describe everything inside a volume is encoded on its boundary, with roughly one bit per Planck area (\( l_p^2 \)). For example, a spherical region of radius \( r \) has a boundary area \( A = 4\pi r^2 \), so its maximum entropy is proportional to \( r^2 \), not \( r^3 \). This is the essence of the Holographic Principle: the physics of a \( d \)-dimensional volume can be described by a theory in \( d-1 \) dimensions on its boundary, much like a hologram projects a 3D image from a 2D film.
The principle gained traction with the development of the AdS/CFT correspondence, proposed by Juan Maldacena in 1997, which provides a concrete realization of holography. AdS/CFT conjectures a duality between a gravitational theory in ( d )-dimensional Anti-de Sitter (AdS) space (a universe with a negative cosmological constant) and a conformal field theory (CFT) on its ( (d-1) )-dimensional boundary. For instance, in the most studied case, type IIB string theory in \( AdS_5 \times S^5 \) (a five-dimensional AdS space times a five-dimensional sphere) is dual to a four-dimensional \( \mathcal{N}=4 \) supersymmetric Yang-Mills theory on the boundary. The CFT, a quantum field theory without gravity, fully encodes the dynamics of the AdS bulk, including gravity, black holes, and quantum effects. The duality implies that bulk phenomena, like the formation of a black hole, correspond to specific states or operators in the boundary CFT. Mathematically, the partition functions of the two theories are equal:
$$ Z_{\text{AdS}} = Z_{\text{CFT}}, $$
where \( Z = \text{Tr}(e^{-\beta H}) \) is the partition function, \( H \) is the Hamiltonian, and \( \beta = 1/(k T) \). This equivalence allows physicists to study complex gravitational phenomena, like quantum gravity, using well-understood quantum field theories.
The Holographic Principle has far-reaching implications. First, it suggests that gravity, traditionally described by general relativity in the bulk, may be an emergent phenomenon arising from quantum interactions on the boundary. In AdS/CFT, the metric of the AdS space is encoded in the CFT’s correlation functions, and the radial dimension of AdS corresponds to the energy scale in the CFT via the renormalization group flow. This is often expressed through the Ryu-Takayanagi formula, which relates the entanglement entropy \( S_{\text{EE}} \) of a region in the CFT to the area of a minimal surface in the AdS bulk:
$$ S_{\text{EE}} = \frac{\text{Area of minimal surface}}{4 G \hbar}. $$
This formula generalizes the Bekenstein-Hawking entropy to arbitrary regions and highlights the deep connection between quantum entanglement and geometry.
Second, the principle challenges our understanding of spacetime. If a 3D universe can be described by a 2D boundary, the extra dimension may be an illusion, much like a hologram creates the appearance of depth. This raises questions about the fundamental nature of reality: is our universe holographic, with physical laws emerging from a lower-dimensional theory? While AdS/CFT applies to AdS spaces, efforts are underway to extend holography to flat spacetimes (like our universe) or de Sitter spaces, though these are less understood.
Third, the Holographic Principle constrains quantum gravity theories. Any consistent theory must respect the area scaling of entropy, ruling out models where information scales with volume. This has influenced string theory, loop quantum gravity, and other approaches, pushing physicists to rethink locality and causality.
Despite its elegance, the Holographic Principle faces challenges. Outside AdS/CFT, explicit holographic dualities for realistic spacetimes are lacking. The principle also raises philosophical questions: if reality is a hologram, what is the "true" dimensionality of the universe? Moreover, encoding bulk dynamics on a boundary requires non-local interactions, which are hard to reconcile with local field theories.
In conclusion, the Holographic Principle is a cornerstone of modern theoretical physics, bridging black hole thermodynamics, quantum mechanics, and gravity. By suggesting that the universe’s information is encoded on a lower-dimensional boundary, it offers a path to unify quantum mechanics and general relativity. Equations like the Bekenstein-Hawking entropy, Bekenstein bound, and Ryu-Takayanagi formula quantify this idea, while AdS/CFT provides a concrete framework. As research progresses, the principle may unlock deeper truths about the universe, perhaps revealing that reality, like a hologram, is a projection of a more fundamental truth.


\subsection{AdS$_n$-CFT$_{n-1}$ Duality}

\subsubsection{AdS$_n$ spacetime}


\subsubsection{CFT$_{n-1}$}

\subsubsection{AdS/CFT correspondence}

AdS/CFT Duality Explanation
The AdS/CFT correspondence, a cornerstone of modern theoretical physics, posits a duality between a conformal field theory (CFT) in ( d )-dimensional Minkowski space and a gravitational theory in ( (d+1) )-dimensional Anti-de Sitter (AdS) space. Introduced by Maldacena, this holographic principle suggests that a strongly coupled CFT, such as \( \mathcal{N}=4 \) super Yang-Mills (SYM) in four dimensions, is equivalent to a weakly coupled supergravity theory in \( AdS_5 \times S^5 \). The correspondence leverages the isometry group \( SO(2,4) \) of \( AdS_5 \), which matches the conformal group of the CFT, enabling a dictionary between CFT operators and AdS fields. For instance, a CFT operator \( O(x) \) with conformal dimension \( \Delta \) couples to a bulk field \( \phi(x, u) \) via boundary interactions, described by the action term \( \int d^4 x , \phi_0(x) O(x) \), where \( \phi_0(x) \) is the boundary value of \( \phi \). Correlation functions in the CFT are computed using the bulk partition function, approximated classically as \[Z_{\text{AdS}}[\phi_0] = e^{-S_{\text{AdS}}[\phi]} \approx \langle e^{\int \phi_0 O} \rangle_{\text{CFT}}\] The large ( N ) limit of the gauge theory, with ’t Hooft coupling \( \lambda = g_{\text{YM}}^2 N \), corresponds to the classical supergravity limit when \( \lambda \gg 1 \), while the string coupling \( g_s \sim 1/N \). This duality allows quantum effects in the CFT, like correlation functions \[\langle O(x_1) \cdots O(x_n) \rangle\] to be computed via classical gravitational dynamics, providing insights into strongly coupled systems, confinement, and even condensed matter physics, by mapping complex quantum phenomena to tractable geometric problems in AdS space.

\subsection{$AdS_2$-CFT$_1$ correspondence}

\subsubsection{2D gravity models - JT gravity}
Jackiw-Teitelboim (JT) gravity is a model of two-dimensional (2D) gravity that provides a valuable framework for understanding quantum gravity and black hole thermodynamics in a simplified setting. Unlike in four dimensions where Einstein's equations produce dynamic degrees of freedom for the metric, in 2D spacetime the Einstein tensor vanishes identically due to topological constraints. Specifically, for any 2D metric $g_{\mu\nu}$, the Einstein tensor $G_{\mu\nu} = R_{\mu\nu} - \frac{1}{2}g_{\mu\nu}R$ is identically zero because the Ricci tensor $R_{\mu\nu}$ is completely determined by the scalar curvature $R$, and the variation of the Einstein-Hilbert action yields no dynamics. Therefore, to have a nontrivial theory of gravity in 2D, auxiliary fields such as the *dilaton* $\phi$ are introduced.

In JT gravity, the action is constructed as

$$
S_{\text{JT}} = \int d^2x \sqrt{-g} \, \phi (R + 2\Lambda),
$$

where $\phi$ is the dilaton field, $R$ is the Ricci scalar of the 2D spacetime metric $g_{\mu\nu}$, and $\Lambda$ is a cosmological constant (often negative, corresponding to an AdS$_2$ background). This action yields second-order field equations for the metric and first-order equations for the dilaton, making the theory solvable. Varying the action with respect to $\phi$ gives

$$
R + 2\Lambda = 0,
$$

which fixes the geometry to a constant curvature spacetime—typically anti-de Sitter (AdS$_2$) for $\Lambda < 0$. Varying the action with respect to the metric gives another equation involving derivatives of $\phi$, which governs how the dilaton profiles across the spacetime.

A key feature of JT gravity is that, despite having no propagating degrees of freedom in the metric, it supports black hole solutions, and its boundary dynamics are nontrivial. These dynamics are governed by the Schwarzian action, which appears when considering the low-energy limit of the boundary mode:

$$
S_{\text{Sch}}[f] = -C \int dt \, \{ f(t), t \},
$$

where $\{f(t), t\}$ is the Schwarzian derivative of the boundary reparametrization $f(t)$, and $C \propto \phi_r$ is related to the value of the dilaton at the boundary. This boundary action captures the low-energy dynamics of near-extremal black holes in higher-dimensional theories and connects JT gravity to the SYK model—a disordered quantum mechanical system with similar infrared behavior.

The JT model also arises from dimensional reduction of higher-dimensional gravity theories. For instance, spherically reducing four-dimensional Einstein gravity under the ansatz

$$
ds^2_{(4)} = g_{\mu\nu}(x) dx^\mu dx^\nu + \Phi^2(x) d\Omega_2^2
$$

produces an effective 2D theory for $g_{\mu\nu}$ and the scalar $\Phi$, leading to actions of the form

$$
S = \int d^2x \sqrt{-g} \left[ \Phi^2 R - \frac{1}{2} (\nabla \Phi)^2 - V(\Phi) \right],
$$

which includes the JT gravity model in specific limits where $\Phi \sim \phi$, and kinetic terms may be neglected.

The beauty of JT gravity lies in its exact solvability and the ability to study non-perturbative aspects of quantum gravity, black hole entropy, and holography in a tractable setting. The model encapsulates the essence of diffeomorphism invariance, black hole thermodynamics, and quantum effects such as Hawking radiation, despite its apparent simplicity.


\subsubsection{1D CFT model - SYK model}
The Sachdev-Ye-Kitaev (SYK) model is a quantum mechanical system of $N$ Majorana fermions $\chi_i(\tau)$ with all-to-all random $q$-body interactions, most commonly $q = 4$. Its action is given by:

$$
S = \int d\tau \left[ \frac{1}{2} \sum_{i=1}^{N} \chi_i \partial_\tau \chi_i - \frac{1}{4!} \sum_{i,j,k,l} J_{ijkl} \chi_i \chi_j \chi_k \chi_l \right],
$$

where $J_{ijkl}$ are real, antisymmetric, Gaussian-random couplings with zero mean and variance

$$
\langle J_{ijkl}^2 \rangle = \frac{3! J^2}{N^3}.
$$

At large $N$, the dominant Feynman diagrams contributing to the two-point function are “melon” diagrams, which can be resummed via a Schwinger-Dyson equation involving the full two-point function $G(\tau) = \frac{1}{N} \sum_i \langle \chi_i(\tau) \chi_i(0) \rangle$ and its self-energy $\Sigma(\tau)$:

$$
\Sigma(\tau) = J^2 G(\tau)^3, \qquad G(i\omega)^{-1} = -i\omega - \Sigma(i\omega).
$$

In the infrared (IR) limit, where $J|\tau| \gg 1$, the kinetic term $\partial_\tau$ becomes negligible and the equations become conformally invariant. The solution to the Schwinger-Dyson equations in this regime is:

$$
G(\tau) = b \frac{\text{sgn}(\tau)}{|J \tau|^{2\Delta}}, \quad \Delta = \frac{1}{4}, \quad b^4 = \frac{1}{4\pi}.
$$

This emergent conformal symmetry is spontaneously and explicitly broken, leading to the appearance of a soft mode governed by the Schwarzian action. By considering reparametrizations $\tau \to f(\tau)$, one finds that the leading IR effective action for these modes is:

$$
S_{\text{Sch}} = -\alpha_S N \int d\tau \, \text{Sch}(f(\tau), \tau), \quad \text{where} \quad \text{Sch}(f(\tau), \tau) = \frac{f'''(\tau)}{f'(\tau)} - \frac{3}{2} \left( \frac{f''(\tau)}{f'(\tau)} \right)^2.
$$

This Schwarzian action governs the breaking of reparametrization symmetry from $\text{Diff}(R)$ to $SL(2, \mathbb{R})$, and it plays a central role in the model's connection to two-dimensional dilaton gravity and holography.

The four-point function is given by a sum over ladder diagrams built from full propagators and a kernel:

$$
K(\tau_1, \tau_2; \tau_3, \tau_4) = -J^2 (q-1) G(\tau_{13}) G(\tau_{24}) G(\tau_{34})^{q-2}.
$$

This kernel acts on bilocal functions, and its eigenfunctions are the conformal partial waves $\mathcal{F}_h(\tau_i)$ of the $SL(2, \mathbb{R})$ group. The resulting four-point function in the conformal limit takes the form:

$$
\langle \chi_i(\tau_1) \chi_i(\tau_2) \chi_j(\tau_3) \chi_j(\tau_4) \rangle = G(\tau_{12})G(\tau_{34}) + \frac{1}{N} \int \frac{dh}{2\pi i} \rho(h) \mathcal{F}_h(\tau_i),
$$

with

$$
\rho(h) = \mu(h) \frac{k(h)}{1 - k(h)},
$$

where $k(h)$ are the eigenvalues of the kernel and $\mu(h)$ is a measure factor involving gamma functions. The poles of $\rho(h)$, given by $k(h) = 1$, determine the dimensions $h$ of bilinear operators exchanged in the four-point function, such as $\mathcal{O}_h = \sum_i \chi_i \partial_\tau^{2n+1} \chi_i$.

Of particular importance is the $h = 2$ mode, which corresponds to the Schwarzian sector. It leads to a divergence in the conformal four-point function and signals the need to include the full effective action, incorporating non-conformal corrections. The dominance of the $h = 2$ exchange in out-of-time-ordered correlators implies maximal quantum chaos, with a Lyapunov exponent:

$$
\lambda_L = \frac{2\pi}{\beta},
$$

which saturates the bound on chaos in quantum systems and matches the behavior of black holes in Einstein gravity. This profound connection makes the SYK model a valuable tool in exploring the AdS/CFT correspondence and quantum aspects of gravity.

\subsection{Virosoro Algebra and the Central Charge}
\label{subsection:central-charge}
The Witt algebra and the Virasoro algebra are fundamental structures in theoretical physics, particularly in conformal field theory (CFT), which describes systems with conformal symmetry, such as those in string theory or critical phenomena in statistical mechanics. The Witt algebra is an infinite-dimensional Lie algebra that captures the infinitesimal conformal transformations of the complex plane or the circle (S¹). Physically, it represents the symmetries of a system that remain invariant under angle-preserving transformations, like stretching or rotating parts of a plane while preserving its local structure. Its basis elements, labeled \(L_n\), correspond to vector fields that generate these transformations, with a Lie bracket \([L_n, L_m] = (n - m)L_{n+m}\), reflecting how these symmetries compose.

The Virasoro algebra, however, is the central player in CFT, extending the Witt algebra to account for quantum effects. In classical systems, symmetries like those in the Witt algebra are straightforward, but quantization introduces anomalies—quantum corrections that modify the symmetry algebra. The Virasoro algebra is the Witt algebra's *central extension*, meaning it adds a new element, a central charge \(Z\), to the algebra, which doesn’t transform under the symmetries but affects their composition. The modified Lie bracket becomes \([L_n, L_m] = (n - m)L_{n+m} + \delta_{n+m,0} \frac{n(n^2 - 1)}{12} Z\), where the extra term, proportional to \(Z\), appears only when \(n + m = 0\). This central charge \(Z\) is a scalar that commutes with all \(L_n\) (\([L_n, Z] = 0\)), making it “central” to the algebra.

The transition from the Witt to the Virasoro algebra is like upgrading a classical machine to a quantum one: the Witt algebra handles the smooth, classical symmetries, but the Virasoro algebra, with its central charge, captures the richer, quantum-deformed structure. This extension is unique (up to equivalence), making the Virasoro algebra the universal framework for describing conformal symmetries in quantum physics, with the central charge acting as a fingerprint of the system’s quantum nature.

We have adopted the following approach to compute the central-charge. From the twisted stress-energy tensor, we can compute the central charge using the following formula:

\begin{equation}
    \label{eqn:central-charge-transformation-eqn}
    \delta_{\epsilon}T_{zz}(z) = \frac{c}{12}\partial^3_z\eta(z) + 2\partial_z\eta(z)T_{zz}(z) + \eta(z)\partial_zT_{zz}(z)
\end{equation}

\section{Motivation}

{\color{red} Central charge calculation ...}

%\section{Organization of the Thesis}
%
%This is how acronym is added \ac{EBSD}. The present dissertation is divided into eight chapters, each of which is further divided into well-structured sections and subsections. \autoref{Chapter1} explains the rationale behind the dissertation and its objectives. \autoref{Chapter2} says something more. 
%

% Chapter Template

\chapter{Literature Review}

\label{Chapter2}

\lhead{Chapter 2. \emph{Literature Review}} 

\section{$1+1D$ gravity coupled to constant EM field strength}
\label{castro-paper}

\lipsum[2-4]

\section{$1+1D$ gravity coupled to the ModMax EM field}
\label{drc-hemant-paper}

\lipsum[2-4]


\section{Constitution of the W-Ni-Fe System}

\lipsum[2-4]

\begin{figure}
    \centering
    \includegraphics[width=0.5\textwidth]{Pictures/Figure 1.png}
    \caption{This is the caption for figure 1.}
    \label{figure:chap2_Figure_2}
\end{figure}



 
% Chapter Template

\chapter{Materials and Methods} % Main chapter title

\label{Chapter3} % Change X to a consecutive number; for referencing this chapter elsewhere, use \ref{ChapterX}

\lhead{Chapter 3. \emph{Materials and Methods}} % Change X to a consecutive number; this is for the header on each page - perhaps a shortened title

\lipsum[2-4]


\section{Starting Material}

\subsection{Tungsten Powder}

\lipsum[2-4]


%% Chapter Template

\chapter{Title of Chapter Four}

\label{Chapter4}

\lhead{Chapter 4. \emph{Title of Chapter Four}} 

\section{Background}

\lipsum[2-4]

\section{Experimental Results}
\subsection{Initial microstructure and texture}

\lipsum[2-4]

\begin{table}
    \centering
        \begin{tabular}{ccc}
        \hline 
        Sample & Grain Size of tungsten $(\mu m)$ & Phase Fraction of Matrix \\
        \hline
         80W-20M & 21 $\pm$ 8 & 0.30 \\
         85W-15M & 20.7 $\pm$ 8.4 & 0.23 \\
         90W-10M & 22 $\pm$ 7.2 & 0.19 \\
         95W-5M & 20.2 $\pm$ 7.3 & 0.13 \\
         98W-2M & 20 $\pm$ 5 & 0.08 \\
        \hline
        \end{tabular}
    \caption{Grain size and phase fraction of sintered WHA.}
    \label{Table:grain_size_and_phase_fraction_of_sintered}
\end{table}

\lipsum[2-4]

\section{Texture evolution during uniaxial compression}
\label{Texture evolution during uniaxial compression}

\lipsum[2-4]

% \begin{figure}
%     \centering
%     \includegraphics[width=\textwidth,keepaspectratio]{Pictures/Chap4_fig_true_stress_true_strain_and_work_hardening.png}
%     \caption{(a) True stress-true strain plot of WHA (b) Work hardening rate vs $\sigma -{\sigma }_0$ of WHA (c) Vicker hardness number of compressed samples.}
%     \label{Fig4.2:Chap4_true_stress_true_strain_and_work_hardening}
% \end{figure}


\section{Summary}

\lipsum[1]

\begin{enumerate}
    \item \lipsum[1]
    
    \item \lipsum[1]
    
    \item \lipsum[1]
    
\end{enumerate}

 

%----------------------------------------------------------------------------------------
%	THESIS CONTENT - APPENDICES
%----------------------------------------------------------------------------------------

\addtocontents{toc}{\vspace{2em}} % Add a gap in the Contents, for aesthetics

\appendix % Cue to tell LaTeX that the following 'chapters' are Appendices

% Include the appendices of the thesis as separate files from the Appendices folder
% Uncomment the lines as you write the Appendices

% Appendix Template

\chapter{Kaluza-Klein reduction scheme} % Main appendix title
%\addtocontents{toc}

\label{AppendixA} % Change X to a consecutive letter; for referencing this appendix elsewhere, use \ref{AppendixX}

\lhead{Appendix A. \emph{Kaluza-Klein reduction}} % Change X to a consecutive letter; this is for the header on each page - perhaps a shortened title

The Kaluza-Klein reduction scheme is a method of dimensional reduction that allows us to obtain lower-dimensional theories from higher-dimensional ones. This method is particularly useful in the context of string theory and other theories of quantum gravity, where we often work in higher dimensions.


\section*{Overview of Kaluza-Klein Reduction}
Kaluza-Klein (KK) reduction is a powerful method in theoretical physics for deriving effective lower-dimensional field theories from higher-dimensional theories. It involves compactifying extra spatial dimensions, often on a circle \( S^1 \), and studying the implications in the resulting lower-dimensional spacetime. The original idea by Kaluza and Klein aimed to unify gravity and electromagnetism by considering a five-dimensional spacetime.

\section*{KK Reduction from \((D+1)\)-Dimensions to \(D\)-Dimensions}
Consider the Einstein-Hilbert action in \( D+1 \) dimensions:
\begin{equation}
\mathcal{L} = \sqrt{-\hat{g}} \, \hat{R}
\end{equation}
where \( \hat{g}_{MN} \) is the metric and \( \hat{R} \) is the Ricci scalar in \( D+1 \) dimensions. To reduce to \( D \) dimensions, compactify one spatial coordinate \( z \sim z + 2\pi L \) on a circle \( S^1 \), and assume all fields are independent of \( z \) (zero-mode truncation). The metric ansatz is
\begin{equation}
d\hat{s}^2 = e^{2\alpha \phi} g_{\mu\nu} dx^\mu dx^\nu + e^{2\beta \phi} (dz + A_\mu dx^\mu)^2
\end{equation}
where \( g_{\mu\nu} \) is the \( D \)-dimensional metric, \( A_\mu \) is a gauge field, and \( \phi \) is a scalar (the dilaton). The constants \( \alpha \) and \( \beta \) are chosen as
\begin{equation}
\alpha^2 = \frac{1}{2(D-1)(D-2)}, \qquad \beta = -(D-2)\alpha
\end{equation}

The components of the higher-dimensional metric in terms of \( D \)-dimensional fields become:
\begin{align}
\hat{g}_{\mu\nu} &= e^{2\alpha \phi} g_{\mu\nu} + e^{2\beta \phi} A_\mu A_\nu \\
\hat{g}_{\mu z} &= e^{2\beta \phi} A_\mu \\
\hat{g}_{zz} &= e^{2\beta \phi}
\end{align}

\section*{Effective Action in \( D \) Dimensions}
After dimensional reduction, the Einstein-Hilbert Lagrangian reduces to:
\begin{equation}
\mathcal{L} = \sqrt{-g} \left( R - \frac{1}{2} (\partial \phi)^2 - \frac{1}{4} e^{-2(D-1)\alpha \phi} F_{\mu\nu} F^{\mu\nu} \right)
\end{equation}
where \( F_{\mu\nu} = \partial_\mu A_\nu - \partial_\nu A_\mu \) is the field strength of \( A_\mu \).

\section*{Reduction of \( n \)-Form Fields}

For a higher-dimensional \( n \)-form field strength \( \hat{F}_{(n)} = d\hat{A}_{(n-1)} \), the reduction yields:
\begin{equation}
\hat{A}_{(n-1)}(x,z) = A_{(n-1)}(x) + A_{(n-2)}(x) \wedge dz
\end{equation}
and thus:
\begin{equation}
\hat{F}_{(n)} = F_{(n)} + F_{(n-1)} \wedge (dz + A)
\end{equation}
with:
\begin{align}
F_{(n)} &= dA_{(n-1)} - dA_{(n-2)} \wedge A \\
F_{(n-1)} &= dA_{(n-2)}
\end{align}

The kinetic terms reduce to:
\begin{equation}
\mathcal{L} = -\frac{1}{2n!} \sqrt{-g} \left( e^{-2(n-1)\alpha \phi} F_{(n)}^2 + e^{2(D-n)\alpha \phi} F_{(n-1)}^2 \right)
\end{equation}

\section*{Symmetries}

The lower-dimensional theory exhibits:
\begin{itemize}
  \item D-dimensional general coordinate invariance,
  \item U(1) gauge symmetry from \( A_\mu \),
  \item A shift symmetry of \( \phi \) accompanied by a rescaling of \( A_\mu \), derived from a global scaling symmetry in the higher-dimensional theory.
\end{itemize}

Kaluza-Klein reduction illustrates how gauge fields and scalar fields emerge from higher-dimensional gravity upon compactification. It serves as a foundational mechanism in string theory and M-theory to relate higher-dimensional unified theories to our observed four-dimensional universe.


%The plastic deformation in metallic materials is of a heterogeneous nature, with strain distributed non-uniformly at the inter and intragranular levels. In order to evaluate the capabilities of full-field models for predicting the deformation arrangement, experimental techniques are required to measure the spatially resolved deformation formation in microstructures.
%
%\section{Electron Backscatter Diffraction}
%\label{Electron Backscatter Diffraction}
%
%Electron Backscatter Diffraction (EBSD) is a technique for measuring crystal orientation using a scanning electron microscope (SEM). A focussed electron beam illuminates a spot on the surface of the sample that is inclined by $70^{\circ}$  to the beam direction. Part of this primary beam is elastically scattered from the crystal lattice. These backscattered electrons diffract through the crystal lattice as they escape the material, producing a spatial electron intensity pattern that indicates the crystal structure and orientation. The diffracted electors emanate spherically from the illuminated point on the sample surface. Since the sample is tilted, a detector can be positioned to collect a portion of the diffraction pattern on a flat plane. The collected Kikuchi pattern (named for a pioneer in electron diffraction) can then be analysed to assess the orientation of the crystal by measuring the position and orientation of bands of constructive interference \cite{maitland2007electron}.
%
%\noindent A crystal orientation map can be collected by rastering the beam across the sample surface, collecting and analysing a Kikuchi pattern at each point. These maps contain information about grain size, grain boundary character, sample texture and alloy phase. EBSD can also be used to assess local deformations \cite{wright2011review}. This is achieved either by direct measurement of elastic strains through cross-correlation of individual Kikuchi patterns \cite{wilkinson2006high,wilkinson2006high,britton2012high} or by quantifying small changes and orientation gradients that are generated by plastic deformation \cite{kamaya2004measurement,kamaya2007local,githinji2013ebsd}.
%
%\subsection{Measures of misorientation}
%\label{Measures of misorientation}
%
%Misorientation is the difference between two orientations and can be expressed as the minimum angle of rotation around any axis that transforms one orientation into another. The plastic deformation of a crystal due to dislocation slip induces a rotation of the crystal lattice. In an unrestrained single crystal, this leads to a net rotation of the material. However, in order to ensure that the grains remain compatible with their neighbours in a polycrystal, deformation gradients and differences in active slip system within single grains are required \cite{barbe2001intergranular}. This leads to orientation gradients that develop within the grains. Each active slip system induces rotation around a different axis, and the amount of rotation is proportional to the amount of slip. The changes in orientation are accommodated in the crystal lattice by means of geometrically necessary dislocations (GNDs), which cause a net rotation of the lattice. A GND density can be calculated from the rotation gradients of the lattice \cite{pantleon2008resolving}.
%
%\noindent The measurement parameters for misorientation define between which orientations the misorientation is calculated and how this relates to the crystal rotation. In a class of misorientation parameters, a misorientation between neighbouring spatial points of an EBSD is calculated. This could be a single point in a single direction or, more generally, in a kernel of positions surrounding each point. The average of the misorientation at each point in the kernel results in the kernel-averaged misorientation (KAM) \cite{wright2011review}. The KAM quantifies local fluctuations in orientation. Large values indicate a large local orientation gradient resulting from a change in the active slip system or an abrupt change in slip activity. Examples of KAM maps taken from a plastically deformed sample are shown in figure \ref{fig:Measures of misorientation1}a-b. The EBSD maps are taken at two different EBSD step sizes and show that KAM values are sensitive to the chosen EBSD step size.
%
%\begin{figure}
%    \centering
%    \includegraphics[scale=1.5]{Pictures/KAM.jpg}
%    \caption{Examples of misorientation parameters for the same region of material. a) and b) are KAM calculated for a 3x3 pixel $^{2}$ kernel for EBSD step sizes of $0.2 \mu \mathrm{m}$ and $0.4 \mu \mathrm{m} .$ c) is ROD calculated to each grains average orientation and d) ROD calculated to the point in each grain with minimum KAM. Taken from \cite{wright2011review}.}
%    \label{fig:Measures of misorientation1}
%\end{figure}
%
%KAM does not indicate gradual changes of orientation relating to gradients of deformation. This can be measured using a second class of parameters, the reference orientation deviation (ROD) \cite{wright2011review}. As the name suggests ROD is the misorientation between each point of a map to a reference orientation.A reference orientation must be defined separately for each grain in the EBSD map, since the misorientation between two grains is generally greater than any gradient in it. Common choices for the reference orientation are the mean orientation of each grain or the point in each grain with the minimum KAM value. Figure \ref{fig:Measures of misorientation1}c-d shows a ROD map for each of these reference orientation choices. Similar gradients can be seen in each map but with different magnitudes of the ROD values, whereby the choice of the reference orientation only changes the zero point of the misorientation. Since ROD measures misorientation relative to a fixed reference, it is not sensitive to step size changes.
%
%\noindent The use of misorientation can indicate intense strain in the grains of a microstructure \cite{kamaya2006quantification}, but cannot identify magnitudes of strains. The distribution of the intragranular plastic strain is also not well predicted due to misorientation compared to local strain measurements from image correlation techniques \cite{kamaya2007local}. The misorientation, however, provides a good measure of the microstructure evolution induced by plastic deformation.

%\input{Appendices/AppendixB}
%\input{Appendices/AppendixC}

\addtocontents{toc}{\vspace{1em}} % Add a gap in the Contents, for aesthetics

\backmatter

%----------------------------------------------------------------------------------------
%	BIBLIOGRAPHY
%----------------------------------------------------------------------------------------
\nocite{*}
\label{Bibliography}

\lhead{\emph{Bibliography}} % Change the page header to say "Bibliography"

\bibliographystyle{ieeetr} % Use the "custom" BibTeX style for formatting the Bibliography

\bibliography{Bibliography} % The references (bibliography) information are stored in the file named "Bibliography.bib"

\end{document}  
